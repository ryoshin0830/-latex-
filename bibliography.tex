% biblatexの設定
\usepackage[backend=biber,    % バックエンドとしてbiberを使用
    style=authoryear-ibid,    % 著者-年方式で、同じ文献の2回目以降の引用ではibidを使用
    citestyle=authoryear-comp,% 引用スタイルは著者-年方式で、複数引用を圧縮表示
    maxcitenames=2,           % 引用時の最大著者数を2名に制限
    minnames=1,               % 最小著者表示数を1名に設定
    maxnames=999,             % 参考文献リストでの最大著者表示数を999名に設定
    uniquelist=false,         % 著者リストの一意性を要求しない
    sorting=none,             % 文献の並び替えを行わない
    defernumbers=true,        % 文献番号の割り当てを遅延させる
    labelnumber=true]{biblatex}  % 文献ラベルに番号を使用

% 日本語文献と非日本語文献を分けるためのフィルター定義
\defbibcheck{japanese}{%      % 日本語文献のみを表示するフィルター
  \iffieldequalstr{langid}{japanese}
    {}
    {\skipentry}%
}

\defbibcheck{notjapanese}{%  % 非日本語文献のみを表示するフィルター
  \iffieldequalstr{langid}{japanese}
    {\skipentry}
    {}%
}

% 文献の並び順を制御するためのマッピング
\DeclareSourcemap{
  \maps[datatype=bibtex]{
    \map{
      \step[fieldsource=langid,match=japanese,final]    % 日本語文献を先に
      \step[fieldset=presort,fieldvalue={a}]
    }
    \map{
      \step[fieldsource=langid,notmatch=japanese,final] % 非日本語文献を後に
      \step[fieldset=presort,fieldvalue={b}]
    }
  }
}

% 参考文献の詳細な並び順の設定
\DeclareSortingTemplate{none}{
  \sort{
    \field{presort}          % まずpresortフィールドで並び替え
  }
  \sort[locale=ja]{
    \field{sortname}
    \field{author}
    \field{editor}
    \field{translator}
    \field{sorttitle}
    \field{title}
  }
  \sort{
    \field{sortyear}         % 最後に年で並び替え
    \field{year}
  }
}

% 参考文献データベースの指定
\addbibresource{マイ・ライブラリ.bib} 