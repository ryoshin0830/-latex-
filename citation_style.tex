% 言語マッピングの設定
\DeclareLanguageMapping{japanese}{english}

% 文献番号のフォーマット設定
\DeclareFieldFormat{labelnumber}{(#1)}  % 文献番号を括弧付きで表示
\renewbibmacro*{begentry}{\printfield{labelnumber}\addspace}  % 文献の開始時に番号を表示

% 著者名の表示形式設定
\DeclareNameAlias{sortname}{family-given}   % 並び替え用の名前形式
\DeclareNameAlias{labelname}{family-given}  % ラベル用の名前形式

% 日本語著者名の特別な表示形式
\DeclareNameFormat{labelname-japanese}{%
  \ifcase\value{uniquename}%
    \usebibmacro{name:family}
    {\namepartfamily}
    {\namepartgiven}
    {\namepartprefix}
    {\namepartsuffix}%
  \fi
  \usebibmacro{name:andothers}
}

% タイトルの表示形式設定
\DeclareFieldFormat[book,collection]{title}{%  % 書籍・コレクションのタイトル
  \iffieldequalstr{langid}{japanese}{『#1』}{#1}%  % 日本語なら『』、それ以外はそのまま
}
\DeclareFieldFormat[article,incollection]{title}{%  % 論文タイトル
  \iffieldequalstr{langid}{japanese}{「#1」}{#1}%  % 日本語なら「」、それ以外はそのまま
}
\DeclareFieldFormat[periodical]{title}{%  % 定期刊行物のタイトル
  \iffieldequalstr{langid}{japanese}{『#1』}{\mkbibemph{#1}}%  % 日本語なら『』、それ以外は斜体
}
\DeclareFieldFormat[article]{journaltitle}{%  % 雑誌タイトル
  \iffieldequalstr{langid}{japanese}{『#1』}{\mkbibemph{#1}}%  % 日本語なら『』、それ以外は斜体
}
\DeclareFieldFormat[article]{pages}{#1}  % ページ番号の表示形式

% 著者名の表示形式
\DeclareNameFormat{author}{%
  \iffieldequalstr{langid}{japanese}
    {\namepartfamily\space\namepartgiven}  % 日本語は姓名の間にスペース
    {\namepartfamily,\space\namepartgiven}%  % 英語は姓,名の形式
  \ifnumequal{\value{listcount}}{\value{liststop}}
    {}
    {\iffieldequalstr{langid}{japanese}{・}{,\space}}%  % 複数著者の区切り
}

% 日付と年号の表示形式
\DeclareFieldFormat{date}{#1}   % 日付の表示形式
\DeclareFieldFormat{year}{#1}   % 年号の表示形式

% 区切り文字の設定
\renewcommand{\newunitpunct}{%  % 文献要素間の区切り
  \iffieldequalstr{langid}{japanese}{.}{.}%  % 日本語なら全角ピリオド、英語なら半角ピリオド
}
\renewcommand{\labelnamepunct}{\space}  % ラベル名の後のスペース

\renewbibmacro{in:}{}  % "in:"の表示を削除

% 引用時の区切り文字設定
\DeclareDelimFormat{nameyeardelim}{\addspace}  % 著者名と年の間のスペース
\DeclareFieldFormat{biblabeldate}{#1}          % 参考文献での日付ラベルの形式
\DeclareDelimFormat{multicitedelim}{\addsemicolon\space}  % 複数引用の区切り

% 日本語用の区切り文字
\renewcommand{\finalnamedelim}{・}     % 最後の著者名の前の区切り
\renewcommand{\multinamedelim}{・}     % 複数著者名の区切り
\renewcommand{\andothersdelim}{}       % "ほか"の前の区切り
\renewcommand{\postnotedelim}{\addspace}  % 後注の前のスペース

% "et al."を"ほか"に変更
\DefineBibliographyStrings{english}{
    andothers = {ほか}
}

% 引用時の著者名表示形式
\DeclareNameFormat{labelname}{%
  \ifnumgreater{\value{liststop}}{2}    % 著者が3人以上の場合
    {\ifnumequal{\value{listcount}}{1}
       {\namepartfamily\space ほか}  % 第1著者名+"ほか"
       {}}
    {\ifnumequal{\value{listcount}}{1}   % 著者が2人以下の場合
       {\namepartfamily%
        \ifnumequal{\value{liststop}}{2}
          {\iffieldequalstr{langid}{japanese}{\finalnamedelim}{,\space}}
          {}}
       {\ifnumequal{\value{listcount}}{2}
          {\namepartfamily}
          {}}}%
}

% 参考文献リストでの著者名表示形式
\DeclareNameFormat{default}{%
  \iffieldequalstr{langid}{japanese}{%   % 日本語の場合
    \namepartfamily\space\namepartgiven%  % 姓名の間にスペース
  }{%                                    % 英語の場合
    \namepartfamily,\space\namepartgiveni.%  % 姓, 名のイニシャル
    \ifnumequal{\value{listcount}}{\value{liststop}}
      {}
      {\ifnumgreater{\value{listcount}}{1}{\addcomma\space}{}}%
  }%
  \ifnumequal{\value{listcount}}{\value{liststop}}
    {}
    {\iffieldequalstr{langid}{japanese}{・}{\addcomma\space}}%  % 著者間の区切り
}

% 日本語用の引用コマンド定義
\makeatletter
% 本文中の引用(田中(2018)は〜)のコマンド
\newcommand{\jcite}[1]{%
  \citeauthor{#1}(\citeyear{#1})%
}

% 括弧書きの引用(〜である(田中,2018))のコマンド
\newcommand{\jpcite}[1]{%
  (\citeauthor{#1},\citeyear{#1})%
}

% 複数文献の括弧書き引用のコマンド
\newcommand{\jpcites}[1]{%
  (\def\i{1}\@for\@citeb:=#1\do{%    % 複数の文献をループ処理
    \ifnum\i=1\else;\fi%              % 2番目以降の文献の前に;を挿入
    \edef\i{0}%
    \citeauthor{\@citeb},\citeyear{\@citeb}%  % 著者名,年の形式で表示
  })%
}
\makeatother 